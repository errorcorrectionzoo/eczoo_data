% Here is an example of a table that you can export
% as vector graphics.
%
% COMPILATION INSTRUCTIONS:
%
% >   lualatex table_template.tex
% >   dvisvgm --pdf table_template.pdf
%
% (If you're on Mac OS, you might have to do "brew install ghostscript" and add
% the option "--libgs=/usr/local/opt/ghostscript/lib/libgs.dylib" to the dvisvgm
% call)
%
%
% SCROLL DOWN TO EDIT THE TABLE (search for "TABLE STARTS HERE")
%
\documentclass[10pt]{article}

\usepackage[table]{xcolor}

% please compile with lualatex
\usepackage{fontspec}

\setsansfont{Source Sans Pro}
\renewcommand{\familydefault}{\sfdefault}

%
% --- begin custom preamble ---
%
\usepackage{tabularx}
\usepackage{booktabs}

\usepackage{amsmath}
\usepackage{amsthm}

\usepackage{phfparen}
\usepackage{phfqit}

%
% --- end custom preamble ---
%
%
% --- begin set up {preview} ---
%
\usepackage[active,delayed,tightpage]{preview}

\pagestyle{empty}
%
% --- end set up {preview} ---
%

\begin{document}
%
\textwidth=500pt\relax
\hsize=\textwidth\relax
\parindent=0pt\relax
\renewcommand{\arraystretch}{1.2}
\newcolumntype{C}{>{\centering\arraybackslash}X}
%
\begin{preview}%
%
%
% TABLE STARTS HERE --->
%
%
\begin{tabular}{lccc}
% If you use tabularx, try to use \textwidth if possible:
%\begin{tabularx}{\textwidth}{XCCC}
\toprule
% Header
\emph{Code} &
$k$ & $d$ & $w$
\\
% Body
\midrule
% -----
Surface codes &
$O(1)$  & $O(\sqrt{n})$  & $4$
\\ 
Hyperbolic surface codes &
$\Omega(n)$ & $\Omega(\log(n))$ & $O(1)$
\\
\rowcolor{green!25!white}
Homological product codes
& $\Omega(n)$ & $\Omega(n)$ & $O(\sqrt{n})$
\\
\multicolumn{1}{m{4.5cm}}{%
% (use p{2cm} instead of m{2cm} for top vertical alignment)
\cellcolor{red!25!white}%
\raggedright\itshape
Example of a cell whose text content is typeset 
as a paragraph over multiple lines
}
& A & \textbf{B} & \emph{C}
\\
% -----
\bottomrule
\end{tabular}%
%
% <--- TABLE ENDED HERE
%
%
%
\end{preview}%
\end{document}

%%% Local Variables:
%%% mode: lualatex
%%% TeX-master: t
%%% End:
